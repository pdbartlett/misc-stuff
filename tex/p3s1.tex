\documentclass[11pt,a4paper]{article}
\title{Paper 3, Supervision 1}
\author{Sophie Bartlett}
\date {October 2022}
\usepackage{amsmath}
\usepackage{amssymb}
\usepackage{mismath}
\begin{document}
\section*{Q7}
\subsection*{a.}
\begin{align*}
Z &= \frac{6050 - 6000}{150} \approx 0.3333 \\
Probability &= 1 - normsdist(Z) \approx 0.3694
\end{align*}
\subsection*{b.}
\begin{align*}
Z &= \frac{6050 - 6000}{150 / \sqrt{40}} \approx 2.1082 \\
Probability &= 1 - normsdist(Z) \approx 0.0175
\end{align*}
\subsection*{c.}
\begin{align*}
Z &= normsinv(1-0.05/2) \approx 1.9600 \\
Lower bound &= 6000 - \frac{150}{\sqrt{40}}Z \approx 5954 \\
Upper bound &= 6000 + \frac{150}{\sqrt{40}}Z \approx 6046
\end{align*}
\section*{Q8}
\begin{align*}
H_0: \mu &\leq 90 \\
H_1: \mu &> 90 \\
\alpha &= 0.1 \\
t &= \frac{94-90}{22/\sqrt{100}} \approx 1.8182 \\
Pvalue &= tdistrt(|t|, 22-1) \approx 0.0417
\end{align*}
The P-value is less than $\alpha$, so we reject the null hypothesis ($H_0$) and conclude
that selling times have increased.
\section*{Q9}
\subsection*{a.}
\begin{align*}
H_0: \mu &\leq 220 \\
H_1: \mu &> 220 \\
\alpha &= 0.01 \\
\overline{X} &\approx 221.10 \\
s &\approx 46.88 \\
t &= \frac{\overline{X} - 220}{s/\sqrt{105}} \approx 0.2411 \\
Pvalue &= tdistrt(|t|, 105-1) \approx 0.4050
\end{align*}
The P-value is greater than $\alpha$, so we cannot reject the null hypothesis ($H_0$) that
the mean price for the area is less than or equal to \$220k.
\subsection*{b.}
\begin{align*}
H_0: \mu &\geq 15 \\
H_1: \mu &< 15 \\
\alpha &= 0.01 \\
\overline{X} &\approx 14.63 \\
s &\approx 4.85 \\
t &= \frac{\overline{X} - 15}{s/\sqrt{105}} \approx -0.7846 \\
Pvalue &= tdistrt(|t|, 105-1) \approx 0.2172
\end{align*}
The P-value is greater than $\alpha$, so we cannot reject the null hypothesis ($H_0$) that
the mean distance from the centre is greater than or equal to 15 miles.
\section*{Q10}
\subsection*{a.}
\begin{align*}
\mu_1 &= \mu_{regular} \\
\mu_2 &= \mu_{decaff} \\
H_0: \mu_1 - \mu_2 &\geq 0 \\
H_1: \mu_1 - \mu_2 &< 0 \\
\alpha &= 0.01 \\
s_p^2 &= \frac{s_1^2(n_1-1)+s_2^2(n_2-1)}{n_1+n_2-2} \approx 1.5544 \\
t &= \frac{(\overline{X_1}-\overline{X_2})-0}{\sqrt{\frac{s_p^2}{n_1}+\frac{s_p^2}{n_2}}}
  \approx -4.5170 \\
Pvalue &= tdistrt(|t|, 50+20-2) \approx 0.000013
\end{align*}
The P-value is less than $\alpha$, so we can reject the null hypothesis ($H_0$) and conclude
that regular coffee drinkers do drink less than decaffeinated coffee drinkers.
\subsection*{b.}
The above assumes \textit{Homogeneity of Variance}, i.e. the variance in the coffee consumption
of the two populations of coffee drinkers is the same. This is plausible, but not guaranteed,
by ``common sense'' considerations; and also somewhat borne by the sample standard deviations
given.
\section*{Q11}
\begin{align*}
H_0: \mu_D &\leq 0 \\
H_1: \mu_D &> 0 \\
\alpha &= 0.05 \\
t &= \frac{\overline{X_D}-\mu_{\overline{X_D}}}{s_{X_D} / \sqrt{n}} \approx 2.4512 \\
Pvalue &= tdistrt(|t|, 8-1) \approx 0.02321
\end{align*}
The P-value is less than $\alpha$, so we can reject the null hypothesis ($H_0$) and conclude
that there are fewer accidents after the changes.
\section*{Q12}
\begin{align*}
H_0: \mu_D &\leq 0 \\
H_1: \mu_D &> 0 \\
\alpha &= 0.1 \\
t &= \frac{\overline{X_D}-\mu_{\overline{X_D}}}{s_{X_D} / \sqrt{n}} \approx 0.4274 \\
Pvalue &= tdistrt(|t|, 5-1) \approx 0.3455
\end{align*}
The P-value is greater than $\alpha$, so we cannot reject the null hypothesis ($H_0$) that that there returns are higher on the NYSE.
\section*{Q13}
\subsection*{a.}
\begin{align*}
\mu_1 &= \mu_{pool} \\
\mu_2 &= \mu_{nopool} \\
H_0: \mu_1 - \mu_2 &= 0 \\
H_1: \mu_1 - \mu_2 &\neq 0 \\
\alpha &= 0.05 \\
\end{align*}
Using Google Sheets \texttt{T.TEST} function gives a P-value of 0.002328,	which is less than
$\alpha$, so we can reject the null hypothesis ($H_0$) and conclude that there is a difference
in price based on having a pool.
\subsection*{b.}
\begin{align*}
\mu_1 &= \mu_{small} \\
\mu_2 &= \mu_{large} \\
H_0: \mu_1 - \mu_2 &= 0 \\
H_1: \mu_1 - \mu_2 &\neq 0 \\
\alpha &= 0.05 \\
\end{align*}
Using Google Sheets \texttt{T.TEST} function gives a P-value of 0.000021,	which is less than
$\alpha$, so we can reject the null hypothesis ($H_0$) and conclude that there is a difference
in price based on property size.
\end{document}
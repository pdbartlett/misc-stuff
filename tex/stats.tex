\documentclass[11pt,a4paper]{article}

\title{Statistics}
\author{Paul D. Bartlett}
\date{October 2022 and onwards}

% Make title and author available
\makeatletter
\let\inserttitle\@title
\let\insertauthor\@author
\makeatother

% Packages
\usepackage{amsmath}
\usepackage{fancyhdr}
\usepackage{float}
\usepackage{mismath}
\usepackage{pgfplots}                                                           
\usepackage{pgf}
\usepackage{tikz}

\pgfplotsset{compat=newest}

% Put title and author in header
\pagestyle{fancy}
\fancyhf{}
\lhead{``\inserttitle''}
\rhead{\insertauthor}
\cfoot{\thepage}
\addtolength{\headheight}{2pt} % space for the rule

% Define a command to add an example plot
\newcommand{\myplot}[1]{
\begin{figure}[H]
\centering
\begin{tikzpicture}[trim axis left]
\begin{axis}[domain=-10:10]
\addplot[mark=none, samples=100, red] function {#1};
\draw[ultra thin, gray] (axis cs:\pgfkeysvalueof{/pgfplots/xmin},0) -- (axis cs:\pgfkeysvalueof{/pgfplots/xmax},0);
\draw[ultra thin, gray] (axis cs:0,\pgfkeysvalueof{/pgfplots/ymin}) -- (axis cs:0,\pgfkeysvalueof{/pgfplots/ymax});
\end{axis}
\end{tikzpicture}
\caption{$f(x) = #1$}
\end{figure}
}

\begin{document}

\maketitle

\section{Mean, variance \& standard deviation}
\subsection{Sigma notation}
For a set of $N$ observations ($x_1, x_2, ..., x_N$) we write the sum of the values as:
\begin{equation*}
\text{sum} = x_1 + x_2 + ... + x_N = \sum_{i=1}^{N}x_i
\end{equation*}
\subsection{Mean}
The mean (usually denoted as $\mu$ or $\overline{x}$) is defined as the sum of the values divided by their count,
and is therefore given by:
\begin{equation*}
\text{mean} = \mu = \overline{x} = \frac{\sum\limits_{i=1}^{N}x_i}{N}
\end{equation*}
\subsection{Variance and standard deviation}
The mean is a measure of central tendency (as are other averages such as median or mode), but it is often also
useful to have a measure of spread. The range (maximum minus minimum) is the simplest such measure, but only
shows total spread of values not how they are distributed. It might therefore be tempting to calculate a ``typical'' spread as the mean
deviation of values from the mean, but unfortunately this always evaluates to zero:
\begin{align*}
\text{deviation from mean} &= \Delta_i = x_i - \overline{x} \\
\text{mean deviation from mean} &= \frac{\sum\limits_{i=1}^{N}\Delta_i}{N} \\
                                &= \frac{\sum\limits_{i=1}^{N}(x_i - \overline{x})}{N} \\
                                &= \frac{\sum\limits_{i=1}^{N}x_i}{N} - \overline{x} \\
                                &= \overline{x} - \overline{x} \\
                                &= 0
\end{align*}
This is because the sum of positive deviations equates to the sum of negative deviations and
the two cancel out. To make all the
deviation measures positive we can square them before summing, so we define variance as the mean squared
deviation from the mean:
\begin{equation*}
\text{variance} = \frac{\sum\limits_{i=1}^{N}(x_i - \overline{x})^2}{N}
\end{equation*}
Unfortunately that means that the units of variance are the square of the units of $x_i$, e.g. if the observations
were of people's height in metres, the variance would be in square metres (area!). We therefore define standard
deviation as the root mean squared deviation from the mean:
\begin{equation*}
\text{standard deviation} = \sigma = \sqrt{\frac{\sum\limits_{i=1}^{N}(x_i - \overline{x})^2}{N}}
\end{equation*}
We can therefore denote variance as $\sigma^2$, and can simplify its calculation as follows:
\begin{align*}
\sigma^2 &= \frac{\sum\limits_{i=1}^{N}(x_i - \overline{x})^2}{N} \\
         &= \frac{\sum\limits_{i=1}^{N}(x_i^2 - 2\overline{x}x_i + \overline{x}^2)}{N} \\
         &= \frac{\sum\limits_{i=1}^{N}x_i^2}{N} - 2\overline{x}\frac{\sum\limits_{i=1}^{N}x_i}{N} + \overline{x}^2 \\
         &= \overline{x^2} - 2\overline{x}^2 + \overline{x}^2 \\
         &= \overline{x^2} - \overline{x}^2
\end{align*}

\section{The Normal Distribution}
\subsection{Derivation}
We know instinctively, or by observation, that many naturally occurring measurements cluster
around a typical value, with higher or lower values becoming less and less frequent as they
become more extreme. What might the mathematical description for such a ``bell-shaped curve''
look like?

Let us consider various functions, $f(x)$, and take a step-wise approach to finding a suitable
one, starting with the simplest possibility: $f(x) = x$:

\myplot{x}

This has two obvious errors: (a) it is not symmetrical, and (b) it goes negative. Both of these
can be solved by using $f(x) = x^2$:

\myplot{x^2}

\end{document}
